\chapter{Introdução}

Linguagens lógicas podem ser utilizadas para verificar propriedades de um
sistema complexo, expressas através de possibilidade, crença, probabilidade
entre outras modalidades. Diferentes modalidades definem diferentes lógicas
modais.

Um teorema é uma fórmula que pode ser provada com base nos axiomas e no conjunto
de regras de inferência que definem um determinado cálculo. Uma vez que um
sistema tenha sido especificado na linguagem lógica por meio de fórmulas, é
possível utilizar métodos de provas para verificá-los.

Em~\cite{nalon} são apresentados cálculos baseados em resolução para quinze
famílias de lógicas modais. As regras de inferência baseiam-se nas propriedades
dos modelos subjacentes, ao invés de se fixar na forma dos axiomas. Deste modo,
obtém-se um procedimento uniforme para se lidar com várias lógicas. Uma das
intenções de tal proposta é justamente prover técnicas que facilitem o projeto
de cálculos combinados tanto para fusões de lógicas quanto para lógicas em que
interações fossem permitidas. Interações são, em geral, caracterizadas por
axiomas contendo operadores das diferentes lógicas componentes.

Grande parte dos provadores para lógicas modais são, porém, baseados em
tradução, o que acaba, por vezes, se tornando inconveniente ao usuário. Além
disso, dada a natureza das aplicações descritas com o auxílio destas lógicas, é
normal o uso da combinação de diferentes linguagens modais. A combinação de
linguagens, todavia, pode acarretar no aumento da complexidade ou mesmo na
indecidibilidade do problema de satisfatibilidade na lógica
resultante~\cite{mdml}. %MANY DIMENSIONAL MODAL LOGICS: plano de trabalho
Portanto, é importante o desenvolvimento de técnicas que possam ser utilizadas
de modo uniforme na combinação de métodos de prova para lógicas obtidas a partir
de fusões e/ou em linguagens que permitam interações.

Os métodos apresentados em~\cite{nalon} e em trabalhos anteriores têm esta
característica de uniformidade, mas carecem de refinamentos a fim de permitir a
construção de ferramentas que possam ser, de fato, utilizadas na verificação
formal de sistemas complexos.

O problema básico de satisfatibilidade da lógica modal K é
PSPACE~\cite{complex_modal}. %THE COMPUTATIONAL COMPLEXITY OF PROV SYSTEM:plano
Entretanto, as complexidades dos algoritmos propostos em~\cite{nalon} ainda não
foram determinadas, %% REVISAR ESTE ITEM%%%
sendo um dos objetos de investigação do atual projeto. Sabe-se, porém, que
métodos de prova para lógica proposicional são
intratáveis~\cite{complex_thproving}. % THE COMPLEXITY OF THE-PROVING:plano
Em geral, métodos baseados em resolução, se ingenuamente implementados, levam
também à utilização exponencial de espaço; entretanto, a utilização de
estratégias garante a linearidade de espaço do método de resolução para lógicas
proposicionais~\cite{toran:prop_resolution}. %NUMBER 9: PLANO
É, portanto, nosso intuito conduzir investigação da extensão e implementação de
estratégias conhecidas (e.g.\ resolução linear, deleção de unidade e subsunção)
que permitam a implementação eficiente dos algoritmos propostos em~\cite{nalon}.

A geração automática de modelos é complementar àquela da prova de teoremas e
realizada em paralelo com a avaliação experimental. Se o provador de teoremas
falha em encontrar uma prova, o modelo automaticamente extraído serve como
testemunha da impossibilidade de se encontrar tal prova. Além disso, com a
possibilidade de uso combinado de estratégias, a não obtenção de um modelo serve
como testemunha da incompletude de tal combinação, sendo portanto ferramenta de
suporte ao entendimento teórico.

O objetivo específico deste trabalho consiste na implementação de um gerador
automático de modelos para a lógica modal proposicional K. A entrada será o
conjunto de cláusulas fornecido pelo provador implementado
em~\cite{silva:impl_provador}. %NUMBER 8: PLANO
A saída será a declaração da inexistência de um modelo, no caso do conjunto de
cláusulas ser insatisfatível, ou a apresentação formal de um modelo que
testemunhe a satisfatibilidade do conjunto de cláusulas.

Os dados obtidos na experimentação prática serão analisados, em conformidade com
o enfoque corrente, de forma comparativa a partir de \textit{benchmarks}
estabelecidos e a partir de \textit{benchmarks} especialmente preparados para
tal fim. Os resultados da experimentação serão publicados em relatório técnico,
que será submetido, na forma de artigo, aos encontros anuais de análise de
performance de provadores de teoremas, em coautoria com os demais estudantes
envolvidos neste projeto. Tais resultados estão presentes neste trabalho na
seção de resultados.











