\chapter{Cálculo}
\label{cap:construcao}

\section{Cláusulas}

\section{Cálculo}
\label{sec:calc}
O cálculo utilizado para implementar o gerador automático de modelos compreende
um conjunto de regras de inferência para lidar com raciocínio tanto
proposicional quando modal. A seguir, denotaremos por $\delta$ o
resultado de unificar os rótulos das premissas de cada regra. Formalmente,
a unificação é dada por uma função $\delta: \mathscr{P}(\mathbb{N})
\longrightarrow \mathbb{N}$, onde $\delta(\{ml\}) = ml$, para qualquer outra
entrada, $\delta$ não está definida. As regras de inferência na
Figura~\ref{tableaux} só podem ser aplicadas se a unificação dos rótulos está
definida\cite{DBLP:conf/tableaux/NalonHD15}.

\begin{figure}[!tbh]
    \centering
    {\footnotesize
        \begin{tabular}{|c|}
            \hline
            \\
                $
                \begin{array}{cll}
                    \mbox{[PROP]} &ml_1: \bigvee^r_{b=1} l_b  \\ 
                                  &ml_2: \\ 
                                  &\ ~\ \vdots \\
                                  &ml_{m}: l' \then \pos{a}~ l \\ \cline{2-2}
                                  & ml+1: l_1 \wedge \ldots \wedge l_m \wedge l \\
                                  & onde\ ml = \delta(\{ml_1,\ldots, ml_m,
                ml_{m+1}\})
                \end{array}
                $
            \\
            \hline
    \end{tabular}}
            \caption{Regras de inferência do tableaux para cláusulas
            proposicionais}
            \label{tableaux}
        \end{figure}


\begin{figure}[!tbh]
    \centering
    {\footnotesize
        \begin{tabular}{|c|}
            \hline
            \\
                $
                \begin{array}{cll}
                    \mbox{[POS]} &ml_1: l_1 ' \then \nec{a}~ l_1 \\ 
                                  &\ ~\ \vdots \\
                                  &ml_m: l_m ' \then \nec{a}~ l_m\\ \cline{2-2}
                                  & ml+1: l_1 \wedge \ldots \wedge l_m \wedge l \\
                                  & onde\ ml = \delta(\{ml_1,\ldots, ml_m,
                ml_{m+1}\})
                \end{array}
                $
            \\
            \hline
    \end{tabular}}
            \caption{Regras de inferência do tableaux para cláusulas modais}
            \label{tableaux}
        \end{figure}

