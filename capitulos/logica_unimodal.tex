Na Lógica Proposicional, existe apenas um contexto no qual se pode raciocinar.
Se uma proposição é verdadeira, não é possível que ela seja falsa. Podemos
estender a noção estudada nesta lógica para uma visão mais ampla, que dependa
dos contextos que se é possível enxergar, ou seja, raciocinar sobre, aumentando o
poder de expressividade da linguagem vista anteriormente. A Lógica Modal define
uma dessas possíveis extensões.

Além das noções expressas através dos operadores proposicionais, a Lógica Modal
estuda argumentos cuja validade depende das noções de ``necessidade'' e
``possibilidade'' (representações análogas são possíveis e foram citadas no
início deste capítulo). Estas noções dizem respeito a valoração das proposições
em diferentes contextos. A notação utilizada para representar estes dois novos
operadores e suas respectivas semânticas na Linguagem Modal estão expressas na Tabela~\ref{tab:op_modal}.

\begin{table}[!h]
    \label{tab:op_modal}
    \caption{Operadores Modais}
\begin{center}
    \begin{tabular}{|rcl|}
       \hline 
       $\pos{} p$ & $:=$ & $p$ é possível, ou seja, $p$ é verdadeiro em algum mundo
       possível\\
       $\nec{} p$ & $:=$ & $p$ é necessário, ou seja, $p$ é verdadeiro em todos os mundos
       possíveis\\
       \hline
    \end{tabular}
\end{center}
\end{table}

As regras de formação de fórmulas são as mesmas da Linguagem Proposicional, com
a seguinte extensão para cobrir os operadores modais:

\begin{definition}
   Se $\varphi$ é uma fórmula bem formada então:
   \begin{itemize}
       \item [$(vii)$] $\pos{} \varphi$ 
       \item [$(viii)$] $\nec{} \varphi$ 
   \end{itemize}
   também são fórmulas bem formadas.
\end{definition}
