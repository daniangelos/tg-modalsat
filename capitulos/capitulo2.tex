\chapter{Revisão Teórica}
\section{Introdução a Lógica Modal}
\label{sec:l_gica_modal}
Neste capítulo, introduziremos a lógica modal através do sistema $S5$. Este
sistema, apesar de ser um dos mais simples possíveis, possui algumas das
principais características do sistema de lógica modal, e essas serão ilustradas
a partir dele. 

O sistema $S5$ é determinado semanticamente por uma avaliação de necessidade e
possibilidade.
Isto é, em resumo %forçado
, a base da estrutura do pensamento modal que data desde o filósofo Leibniz: uma
proposição é \textit{necessária} se ocorre em todos os possíveis mundos 
e \textit{possível} se ocorre em algum mundo~\cite{chellas:modal_logic}.
A ideia é que coisas %revise
diferentes podem ser verdadeiras em mundos diferentes, mas qualquer que ocorra
em todos os possíveis mundos é necessário, enquanto o que ocorre em pelo menos
um mundo é possível.

%As proposições são da forma $\Box A$
%$\Diamond A$

Uma sentença da forma $\Box A$ --- \textit{necessariamente} $A$ --- é verdadeira se,
e somente se, $A$ é uma proposição verdadeira em todos os possíveis mundos; já
uma sentença da forma $\Diamond A$ --- \textit{possivelmente} $A$ --- é verdadeira
no caso em que $A$ é uma proposição verdadeira em algum possível mundo.

Uma ilustração válida é uma coleção de possíveis mundos, incluindo nosso
próprio, o mundo real, onde sentenças da linguagem são possivelmente verdadeiras
ou falsas. O propósito principal da lógica modal é modelar isto, e isto é feito
listando uma sequência possivelmente infinita de conjuntos de possíveis mundos,
\begin{equation}
    P_0,\ P_1,\ P_2,\ldots
\end{equation}

A intuição por trás deste modelo é que, para cada número natural $n$, o conjunto
$P_n$ contém somente os possíveis mundos onde a sentença $\mathbb{P}_n$ é
verdadeira. Em outras palavras, a sequência 
$P_0,\ P_1,\ P_2,\ldots$
ilustra as sentenças at\^omicas estipulando em quais possíveis mundos elas
ocorrem, e.g onde são verdadeiras,
(e, por omissão, em quais mundos elas são falsas):
\begin{equation}
    \mathbb{P}_n\ is\ true\ at\ a\ possible\ world\ \alpha\ if\ and\ only\ if\
    \alpha \in P_n 
\end{equation}

Um modelo no sistema $S5$ é, portanto, uma tupla:
<$W$,~$P$>
onde $W$ é um conjunto de possíveis mundos e $P$ uma abreviação para a sequência
infinita 
$P_0,\ P_1,\ P_2,\ldots$
de subconjuntos de $W$.
Note que $W$ pode conter mundos que não estão presentes em nenhum dos conjuntos
$P_n$; de fato, qualquer um desses conjuntos pode ser vazio.

Definimos o valor da sentença (verdadeiro ou falso) de acordo com sua forma e em
termos de um possível mundo em um modelo.
Usamos a notação:
\begin{equation}
    \label{modal:truth}
    \models ^{\mathcal{M}}_{\alpha} A 
\end{equation}
onde $A$ é uma sentença e $\alpha$ é um possível mundo em um modelo
$\mathcal{M}=<W,P>$.
A lei~\ref{modal:truth} é um resumo para: $A\ is\ true\ at\ \alpha\ in\
\mathcal{M}$.

As condições de verdade estão expressadas na Tabela~\ref{table:truth}.

\begin{center}
    \begin{table}[h!]
\label{table:truth}
    \caption{Truth conditions}

    \begin{tabular}{ll}

        \vspace{2mm}
        (1) & $\models ^{\mathcal{M}}_{\alpha} \mathbb{P}_n$ se e somente se $\alpha \in
        P_n$ com $n=0,1,2,\ldots$\\
        \vspace{2mm}
        (2)  & $\models ^{\mathcal{M}}_{\alpha} $\\
        \vspace{2mm}
        (3)  & $Not\ \models ^{\mathcal{M}}_{\alpha} $\\
        \vspace{2mm}
        (4)  & $\models ^{\mathcal{M}}_{\alpha} \neg A$ se e somente se $not\
        \models ^{\mathcal{M}}_{\alpha} A$\\
        \vspace{2mm}
        (5)  & $\models ^{\mathcal{M}}_{\alpha} $\\
        \vspace{2mm}
        (6)  & $\models ^{\mathcal{M}}_{\alpha} $\\
        \vspace{2mm}
        (7)  & $\models ^{\mathcal{M}}_{\alpha} $\\
        \vspace{2mm}
        (8)  & $\models ^{\mathcal{M}}_{\alpha} $\\
        \vspace{2mm}
        (9)  & $\models ^{\mathcal{M}}_{\alpha} \Box A $\\
        \vspace{2mm}
        (10) & $\models ^{\mathcal{M}}_{\alpha} \Diamond A $\\

    \end{tabular}
\end{table}
\end{center}

Alguns esclarecimentos sobre essas definições podem ser úteis:
\begin{itemize}
    \item A cláusula (1) reflete a premissa sobre os conjuntos $P_0, P_1, P_2,
        \ldots$ em um modelo: uma sentença at\^omica $\mathbb{P}_n$ é verdadeira
        em um possível mundo $\alpha$ somente no caso em que $\alpha$ é um
        elemento do conjunto $P_n$.
    \item De acordo com a cláusula (2), a constante verdadeira $\top$ é sempre
        válida em $\alpha$.
    \item Por (3), a constante falsa $\bot$ é sempre falsa em $\alpha$.
    \item A cláusula (4) afirma que a negação $\neg A$ é verdadeira em $\alpha$
        se, e somente se, sua negação $A$ é falsa em $\alpha$.
    \item A afirmação da cláusula (5) diz que uma conjunção $A \wedge B$ é
        verdadeira em $\alpha$ somente no caso em que ambas sentenças, $A$ e
        $B$ o são.
    \item Já de acordo com a cláusula (6), uma disjunção $A \vee B$ é verdadeira
        em $\alpha$ quando pelo menos uma das sentençãos, $A$ ou $B$ o é.
    \item A inteção com a cláusula (7) é compreender que uma implicação $A \rightarrow
        B$ é verdadeira em $\alpha$ desde que nunca ocorra que a sentença antecedente,
        $A$, seja verdadeira ao mesmo tempo em que a consequente, $B$, seja
        falsa.
    \item Similarmente, na cláusula (8) a intenção se repete de ambos os lados,
        ou seja, a condição $A \leftrightarrow B$ é verdadeira quando ambas
        sentenças, $A$ e $B$, são verdadeiras, ou ambas são falsas.
    \item A cláusua (9) formula a interpretação leibniziana de necessidade:
        $\Box A$
    \item Finalmente, de acordo com a cláusula (10), $\Diamond A$
\end{itemize}

Uma sentença verdadeira em cada mundo possível de todo modelo é chamada
\textit{válida}. Usamos o símbolo $\models$ novamente, mas desta vez sem as
marcações superior e inferior (para enfatizar que a sentença é válida, sem
importar o mundo ou o modelo), e escrevemos $\models A$ quando $A$ é uma
sentença válida.
Mais formalmente, então, definimos a validade de uma sentença da seguinte forma:

\begin{equation}
    \models A\ se,\ e\ somente\ se,\ para\ todo\ modelo\ \mathcal{M}\ e\ para\ 
    qualquer\ mundo\ \alpha\ em\ \mathcal{M},\ temos\ que\ \models ^{\mathcal{M}}_\alpha 
\end{equation}


%% KEEP ON BABY

%??
\section{Preliminares lógicas}
\subsection{Sintaxe}

Esta seção é dedicada a trazer o básico dos conceitos de sintaxe da linguagem da
lógica modal. As definições formais apresentadas podem ser úteis ao
entendimento.

\textbf{Sentenças.} A linguagem é baseada em um conjunto enumerável de sentenças
\textit{at\^omicas}: 
\begin{equation}
    \mathbb{P}_0, \mathbb{P}_1, \mathbb{P}_2, \ldots 
\end{equation}
Estas são as sentenças mais simples possívels.

As \textit{sentenças moleculares} (não-at\^omicas) são formadas por meio das nove
\textit{operações sintáticas}, ou \textit{operadores lógicos}:
\begin{equation}
   \top, \bot, \neg, \wedge, \vee, \rightarrow, \leftrightarrow, \Box, \Diamond
\end{equation}

Como mencionado na seção anterior, $\top$ e $\bot$ são operadores de aridade
zero, ou constantes; $\neg$, $\Box$ e $\Diamond$ são operadores de aridade um; e
$\wedge$, $\vee$, $\rightarrow$ e $\leftrightarrow$ são operadores de aridade
dois.

O conjunto de sentenças pode, então, ser definido formalmente como mostrado na
Tabela~\ref{table:sentences}.

\begin{center}
    \begin{table}[h!]
\label{table:sentences}
    \caption{Truth conditions}

    \begin{tabular}{ll}
        \vspace{2mm}
        (1) & a \\
        \vspace{2mm}
        (2)  & a \\
        \vspace{2mm}
        (3)  & a \\
        \vspace{2mm}
        (4)  &a \\
        \vspace{2mm}
        (5)  & a \\
        \vspace{2mm}
        (6)  &a \\
        \vspace{2mm}
        (7)  &a \\
        \vspace{2mm}
        (8)  &a \\
        \vspace{2mm}
        (9)  &a \\
        \vspace{2mm}
        (10) &a
    \end{tabular}
\end{table}
\end{center}
















