\chapter{Revisão Teórica}
\section{Lógica Modal}
\label{sec:l_gica_modal}
Neste capítulo, introduziremos a lógica modal através do sistema $S5$. Este
sistema, apesar de ser um dos mais simples possíveis, possui algumas das
principais características do sistema de lógica modal, e essas serão ilustradas
a partir dele. 

O sistema $S5$ é determinado semanticamente por uma avaliação de necessidade e
possibilidade.
Isto é, em resumo %forçado
, a base da estrutura do pensamento modal que data desde o filósofo Leibniz: uma
proposição é \textit{necessária} se ocorre em todos os possíveis mundos 
e \textit{possível} se ocorre em algum mundo~\cite{chellas:modal_logic}.
A ideia é que coisas %revise
diferentes podem ser verdadeiras em mundos diferentes, mas qualquer que ocorra
em todos os possíveis mundos é necessário, enquanto o que ocorre em pelo menos
um mundo é possível.

%As proposições são da forma $\Box A$
%$\Diamond A$

Uma sentença da forma $\Box A$ --- \textit{necessariamente} $A$ --- é verdadeira se,
e somente se, $A$ é uma proposição verdadeira em todos os possíveis mundos; já
uma sentença da forma $\Diamond A$ --- \textit{possivelmente} $A$ --- é verdadeira
no caso em que $A$ é uma proposição verdadeira em algum possível mundo.

Uma ilustração válida é uma coleção de possíveis mundos, incluindo nosso
próprio, o mundo real, onde sentenças da linguagem são possivelmente verdadeiras
ou falsas. O propósito principal da lógica modal é modelar isto, e isto é feito
listando uma sequência possivelmente infinita de conjuntos de possíveis mundos,
\begin{equation}
    P_0,\ P_1,\ P_2,\ldots
\end{equation}

A intuição por trás deste modelo é que, para cada número natural $n$, o conjunto
$P_n$ contém somente os possíveis mundos onde a sentença $\mathbb{P}_n$ é
verdadeira. Em outras palavras, a sequência 
$P_0,\ P_1,\ P_2,\ldots$
ilustra as sentenças atômicas estipulando em quais possíveis mundos elas
ocorrem, e.g onde são verdadeiras,
(e, por omissão, em quais mundos elas são falsas):
\begin{equation}
    \mathbb{P}_n\ is\ true\ at\ a\ possible\ world\ \alpha\ if\ and\ only\ if\
    \alpha \in P_n 
\end{equation}

Um modelo no sistema $S5$ é, portanto, uma tupla:
<$W$,~$P$>
onde $W$ é um conjunto de possíveis mundos e $P$ uma abreviação para a sequência
infinita 
$P_0,\ P_1,\ P_2,\ldots$
de subconjuntos de $W$.
Note que $W$ pode conter mundos que não estão presentes em nenhum dos conjuntos
$P_n$; de fato, qualquer um desses conjuntos pode ser vazio.

Definimos o valor da sentença (verdadeiro ou falso) de acordo com sua forma e em
termos de um possível mundo em um modelo.
Usamos a notação:
\begin{equation}
    \label{modal:truth}
    \models ^{\mathcal{M}}_{\alpha} A 
\end{equation}
onde $A$ é uma sentença e $\alpha$ é um possível mundo em um modelo
$\mathcal{M}=<W,P>$.
A lei~\ref{modal:truth} é um resumo para: $A\ is\ true\ at\ \alpha\ in\
\mathcal{M}$.


