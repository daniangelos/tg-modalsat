A lógica permite chegar a uma conclusão (ou mais de uma conclusão) a partir de
um conjunto de premissas (ou hipóteses), isto é, lógica expressa raciocínio. É
necessário entender claramente o que é lógica e por que seu estudo é importante
(e difícil), portanto, esta seção é dedicada a este fim. Nesta seção são
apresentadas algumas definições importantes que são centrais no estudo da
lógica.

Lógica é a análise e a avaliação de argumentos~\cite{gensler}, é a tentativa de
se separar linhas de raciocínios viáveis das inviáveis. É possível definir
raciocínio sobre qualquer tópico, tanto filosófico (por exemplo livre arbitrio e
determinismo, existência de um deus, moralidade etc) como não-filosófico (por
exemplo poluição, futebol, construção de um arranha-céu etc). Sendo assim, a
lógica é uma ferramenta muito útil no raciocínio tanto de questões com um
significado mais profundo, quanto em questões do dia a dia.

Pode-se pensar em três principais justificativas para o estudo de lógica.
Primeiro, é difícil imaginar uma tomada de decisão ou a chegada a uma conclusão,
uma resposta a alguma pergunta, sem pensar que houve uma linha de raciocínio
envolvida, portanto raciocínio é importante. Raciocínio e habilidades analíticas
em geral são fundamentais nas mais diversas áreas. Observe, então, que a lógica
é importante pois seu estudo é a tentativa de compreensão e evolução do
raciocínio. Segundo, a lógica pode aprofundar conhecimentos filosóficos. Sem
lógica, um indivíduo pode apenas divagar de forma vaga sobre as questões de vida
que a filosofia levanta, ou seja, há ausência de suporte ferramental para
compreender e avaliar raciocínios desta natureza. Finalmente, lógica pode ser
bastante divertido. Ela desafia o pensamento a seguir caminhos novos e realizar
processos diferentes, como a montagem de um quebra-cabeça.

A Figura~\ref{fig:exemplo_rac} mostra um exemplo de problema sobre o qual
é possível raciocinar usando lógica.

\begin{figure}[!tbh]
\label{fig:exemplo_rac}
\begin{center}
    \begin{tabular}{cc}
        Caso 1 & Caso 2 \\
        \begin{tabular}{|c|}
            \hline
            Se o cachorro latir, o bebê vai acordar. \\
            O bebê não acordou.\\
            \hline
        \end{tabular}
        &
        \begin{tabular}{|c|}
            \hline
            Se o cachorro latir, o bebê vai acordar. \\
            O cachorro não latiu.\\
            \hline
        \end{tabular}
    \end{tabular}
\end{center}
    \caption{Exemplo de problema sobre o qual se pode raciocinar}
\end{figure}

No primeiro caso, a conclusão de que o cachorro não latiu é trivial. Já no
segundo caso, algumas pessoas se sentiriam tentadas a concluir que o bebê não
acordou, o que não é necessariamente verdade, já que o bebê poderia ter acordado
com outros ruídos ou por estar com fome, por exemplo. A intuição lógica pode ser
trabalhada com a finalidade de servir como ferramenta de raciocínio sobre
argumentos. Por se tratar de um conhecimento acumulativo, quanto mais
aprofundado o estudo de lógica, mais precisa se torna esta ferramenta.

\begin{definition}
   Um \textbf{argumento}, no sentido usado em lógica, é um conjunto de
   afirmações consistindo em premissas e uma conclusão. As \textbf{premissas}
   são afirmações de partida, ou hipóteses, que dão suporte a uma evidência; o
   que estas hipóteses evidenciam é definido como \textbf{conclusão}.
\end{definition}

No exemplo da Figura~\ref{fig:exemplo_arg}, as duas primeiras linhas representam
as premissas do argumento e a última linha é a conclusão.

\begin{figure}[!tbh]
\label{fig:exemplo_arg}
\begin{center}
        \begin{tabular}{|l|}
            \hline
            Se o cachorro latir, o bebê vai acordar. \\
            O bebê não acordou.\\
            Portanto, o cachorro não latiu.\\
            \hline
        \end{tabular}
\end{center}
    \caption{Exemplo de argumento}
\end{figure}

A definição a seguir representa uma primeira noção de um argumento
\textbf{válido}, no contexto de lógica.

\begin{definition}
   Um argumento é \textbf{válido} quando for impossível ou contraditório ter
   todas as premissas verdadeiras e a conclusão falsa.
\end{definition}

O que se quer dizer com essa definição é apenas que, em um argumento válido, a
conclusão \textit{segue} das premissas~\cite{gensler}, ou seja, se as premissas
forem todas verdadeiras, então a conclusão também é.
A Figura~\ref{fig:exemplo_arg} representa um exemplo de argumento válido,
enquanto a Figura~\ref{fig:exemplo_arg_invalido} representa um exemplo de
argumento inválido, como visto anteriormente.

\begin{figure}[!tbh]
\label{fig:exemplo_arg_invalido}
\begin{center}
        \begin{tabular}{|l|}
            \hline
            Se o cachorro latir, o bebê vai acordar. \\
            O cachorro não latiu.\\
            Portanto, o bebê não acordou.\\
            \hline
        \end{tabular}
\end{center}
    \caption{Exemplo de argumento inválido}
\end{figure}

Estas definições iniciais são apenas alguns passos na direção da compreensão da
ideia geral que a lógica trata. Lógica requer estudos cuidadosos, possui 
tópicos difíceis além de parcialmente abstratos.
