%%%%%%%%%%%%%%%%%%%%%%%%%%%%%%%%%%%%%%%%
% Classe do documento
%%%%%%%%%%%%%%%%%%%%%%%%%%%%%%%%%%%%%%%%

% Nós usamos a classe "unb-cic".  Deixe apenas uma das linhas
% abaixo não-comentada, dependendo se você for do bacharelado ou
% da licenciatura.

\documentclass[bacharelado]{unb-cic}
%\documentclass[licenciatura]{unb-cic}

%\ifx\HCode\UnDef\else\hypersetup{tex4ht}\fi

%%%%%%%%%%%%%%%%%%%%%%%%%%%%%%%%%%%%%%%%
% Pacotes importados
%%%%%%%%%%%%%%%%%%%%%%%%%%%%%%%%%%%%%%%%

\usepackage[brazil,american]{babel}
\usepackage[T1]{fontenc}
\usepackage{indentfirst}
\usepackage{listings}
\usepackage{natbib}
\usepackage{xcolor,graphicx,url}
\usepackage[utf8]{inputenc}
\usepackage{mathtools}
\usepackage{mhchem}
\usepackage{tikz}
\usetikzlibrary{trees,er,matrix,mindmap,backgrounds}
\usetikzlibrary{shadings,intersections,calc}
\usetikzlibrary{shapes,arrows,patterns,snakes}
\usepackage{pgfplots}
\usepackage{pgfkeys}
\usepackage{mathrsfs}
\usepgfplotslibrary{units}
\pgfdeclarelayer{nodelayer}
\pgfdeclarelayer{edgelayer}
\pgfsetlayers{edgelayer,nodelayer,main}
%%%<
\usepackage{verbatim}
%%%>

\usepackage{amsmath}
\usepackage{amsthm}
\usepackage{amssymb}
\usepackage{mathpartir}

\tikzstyle{newstyle}=[circle,draw=black,line width=2,minimum size=2.5cm]
\tikzstyle{bolinha}=[circle,draw=black,line width=2,minimum size=1cm]
\tikzstyle{quarequare}=[rectangle,draw=black,line width=2,minimum size=1cm]
\tikzstyle{rect}=[rectangle,draw=black,line width=1,minimum width=10cm,minimum
height=4cm]
\tikzstyle{rect2}=[rectangle,draw=black,line width=1,minimum width=11cm,minimum
height=7cm]
\tikzstyle{int}=[draw, fill=blue!20, minimum size=2em]
\tikzstyle{init} = [pin edge={to-,thin,black}]
%%%%%%%%%%%%%%%%%%%%%%%%%%%%%%%%%%%%%%%%
% Cores dos links
%%%%%%%%%%%%%%%%%%%%%%%%%%%%%%%%%%%%%%%%

% Veja o arquivos cores.tex se quiser ver que outras cores estão
% pré-definidas.  Utilizando o comando \hypersetup abaixo nós
% evitamos aquelas caixas vermelhas feias em volta dos links.

%%%%%%%%%%%%%%%%%%%%%%%%%%%%%%%%%%%%%%%%
% Cores do estilo Tango
%%%%%%%%%%%%%%%%%%%%%%%%%%%%%%%%%%%%%%%%

\definecolor{LightButter}{rgb}{0.98,0.91,0.31}
\definecolor{LightOrange}{rgb}{0.98,0.68,0.24}
\definecolor{LightChocolate}{rgb}{0.91,0.72,0.43}
\definecolor{LightChameleon}{rgb}{0.54,0.88,0.20}
\definecolor{LightSkyBlue}{rgb}{0.45,0.62,0.81}
\definecolor{LightPlum}{rgb}{0.68,0.50,0.66}
\definecolor{LightScarletRed}{rgb}{0.93,0.16,0.16}
\definecolor{Butter}{rgb}{0.93,0.86,0.25}
\definecolor{Orange}{rgb}{0.96,0.47,0.00}
\definecolor{Chocolate}{rgb}{0.75,0.49,0.07}
\definecolor{Chameleon}{rgb}{0.45,0.82,0.09}
\definecolor{SkyBlue}{rgb}{0.20,0.39,0.64}
\definecolor{Plum}{rgb}{0.46,0.31,0.48}
\definecolor{ScarletRed}{rgb}{0.80,0.00,0.00}
\definecolor{DarkButter}{rgb}{0.77,0.62,0.00}
\definecolor{DarkOrange}{rgb}{0.80,0.36,0.00}
\definecolor{DarkChocolate}{rgb}{0.56,0.35,0.01}
\definecolor{DarkChameleon}{rgb}{0.30,0.60,0.02}
\definecolor{DarkSkyBlue}{rgb}{0.12,0.29,0.53}
\definecolor{DarkPlum}{rgb}{0.36,0.21,0.40}
\definecolor{DarkScarletRed}{rgb}{0.64,0.00,0.00}
\definecolor{Aluminium1}{rgb}{0.93,0.93,0.92}
\definecolor{Aluminium2}{rgb}{0.82,0.84,0.81}
\definecolor{Aluminium3}{rgb}{0.73,0.74,0.71}
\definecolor{Aluminium4}{rgb}{0.53,0.54,0.52}
\definecolor{Aluminium5}{rgb}{0.33,0.34,0.32}
\definecolor{Aluminium6}{rgb}{0.18,0.20,0.21}

\hypersetup{colorlinks=true,
  linkcolor=DarkScarletRed,
  citecolor=DarkScarletRed,
  filecolor=DarkScarletRed,
  urlcolor= DarkScarletRed
}

\definecolor{mygreen}{rgb}{0,0.6,0}
\definecolor{mygray}{rgb}{0.5,0.5,0.5}
\definecolor{mymauve}{rgb}{0.58,0,0.82}

\newtheorem{theorem}{Theorem}[section]
\newtheorem{lemma}[theorem]{Lemma}
\newtheorem{proposition}[theorem]{Proposition}
\newtheorem{corollary}[theorem]{Corollary}
\newtheorem{definition2}[theorem]{Def.}

\newenvironment{proof2}[1][Proof]{\begin{trivlist}
\item[\hskip \labelsep {\bfseries #1}]}{\end{trivlist}}
\newenvironment{example}[1][Example]{\begin{trivlist}
\item[\hskip \labelsep {\bfseries #1}]}{\end{trivlist}}
\newenvironment{remark}[1][Remark]{\begin{trivlist}
\item[\hskip \labelsep {\bfseries #1}]}{\end{trivlist}}


\ifx\fmtname\@psfmtname \else \def\cmsy@{2}\fi % make sure we get cmsy
\def\sometime{\mathord{\hbox{\large$\mathchar"0\cmsy@7D$}}}


\newcommand{\always}{\raisebox{-.2ex}{
			   \mbox{\unitlength=0.9ex
			   \begin{picture}(2,2)
			   \linethickness{0.06ex}
			   \put(0,0){\line(1,0){2}}
			   \put(0,2){\line(1,0){2}}
			   \put(0,0){\line(0,1){2}}
			   \put(2,0){\line(0,1){2}}
			   \end{picture}}}
		      \,}

\newcommand{\alwaysi}[1]{\raisebox{-.2ex}{
			   \mbox{\unitlength=0.9ex
			   \begin{picture}(2,2)
			   \linethickness{0.06ex}
			   \put(0,0){\line(1,0){2}}
			   \put(0,2){\line(1,0){2}}
			   \put(0,0){\line(0,1){2}}
			   \put(2,0){\line(0,1){2}}
                           \put(0.5,0.5){\scriptsize$#1$}
			   \end{picture}}}}

\newcommand{\nec}[1]{\!\alwaysi{#1}\,}
\newcommand{\pos}[1]{{	   \mbox{\unitlength=0.9ex
			   \begin{picture}(2,2)
			   \linethickness{0.06ex}
			   \put(0,0){$\sometime$}
                           \put(0.8,0.4){\tiny$#1$}
			   \end{picture}}}\,}

%\newcommand{\pos}[1]{\sometime _{#1}}
\newcommand{\then}{\Rightarrow}
\newcommand{\onlyif}{\Leftarrow}
\newcommand{\ifonlyif}{\Leftrightarrow}
\newcommand{\tvalue}[1]{\mbox{\it #1\/}}
\newcommand{\constant}[1]{\mbox{\rm\bf #1}}
\newcommand{\system}[3]{\raisebox{.2ex}[1.2ex]{\raisebox{-.2ex}{{\sf #1}}{$_{#2}^{#3}$}}}
%\newcommand{\system}[3]{{\sf #1}$_{#2}^{#3}$}
\newcommand{\set}[1]{\mbox{$\mathcal{#1}$}}
\newcommand{\WFF}[2]{{\sf WFF}{\mbox{$_{\mbox{\small\sf #1}_{#2}}$}}}
\newcommand{\know}[1]{\mbox{K$_{#1}\,$}}
\newcommand{\knownot}[1]{\mbox{$\nec{#1}\neg\,$}}
\newcommand{\believe}[1]{\mbox{B$_{#1}\,$}}
\newcommand{\relation}[2]{\mbox{$\mathcal #1$$_{#2}$}}
%\newcommand{\universal}{\always^{*}}
\newcommand{\universal}{\always^*}
\newcommand{\snf}[1]{{\sf SNF}\mbox{$\textstyle _{#1}$}}
\newcommand{\binomial}[2]{\left(\! \begin{array}{c}
                                 #1\\
                                 #2
                                 \end{array}
                          \!\right )}
\newcommand{\ap}[1]{\alpha(#1)}

\newcommand{\ib}{\bf}

\newcommand{\NKN}{\hbox{\it NKN}}
\newcommand{\NEW}{\hbox{\it NEW}}


\newcommand{\Nat}{\mbox{$\mathbb N$}}

\newcommand{\commentts}[1]{}
     

\lstset{language=C,
    basicstyle=\ttfamily,
    numbers=left,      
    keywordstyle=\color{blue}\ttfamily,
    stringstyle=\color{red}\ttfamily,
    commentstyle=\color{mygray}\ttfamily,
    showstringspaces=false,
    morecomment=[l][\color{mymauve}]{\#}
}


%%%%%%%%%%%%%%%%%%%%%%%%%%%%%%%%%%%%%%%%
% Informações sobre a monografia
%%%%%%%%%%%%%%%%%%%%%%%%%%%%%%%%%%%%%%%%

\title{Geração Automática de Modelos em Lógicas Modais: Implementação}

\orientador[a]{\prof[a] \dr[a] Cláudia Nalon}{CIC/UnB}
%\coorientador[a]{\prof[a] \dr[a] Coorientadora}{MAT/UnB}
\coordenador{\prof \dr Rodrigo Bonifácio de Almeida}{CIC/UnB}
\diamesano{6}{junho}{2016}

\membrobanca{\prof \dr Professor I}{CIC/UnB}
\membrobanca{\prof \dr Professor II}{CIC/UnB}

\autor{Daniella Albuquerque}{dos Angelos}
\CDU{004.4}

\palavraschave{lógica modal, método de resolução, geração automática}
\keywords{modal logics, resolution method, automated generation}



%%%%%%%%%%%%%%%%%%%%%%%%%%%%%%%%%%%%%%%%
% Texto
%%%%%%%%%%%%%%%%%%%%%%%%%%%%%%%%%%%%%%%%

\begin{document}
  \maketitle
  \pretextual

  \begin{dedicatoria}
  Dedico a....
  \end{dedicatoria}

  \begin{agradecimentos}
  Agradeço a....
  \end{agradecimentos}

  \begin{resumo}
      A geração automática de modelos é uma técnica complementar àquela da prova
      de teoremas e realizada em paralelo com a avaliação experimental. Se o
      provador de teoremas falha em encontrar uma prova, o modelo
      automaticamente extraído serve como testemunha da impossibilidade de se
      encontrar tal prova. Além disso, com a possibilidade de uso combinado de
      estratégias, a não obtenção de um modelo serve como testemunha da
      incompletude de tal combinação, sendo portanto ferramenta de suporte ao
      entendimento teórico. Este trabalho tem como objetivo a implementação de
      um gerador automático de modelos para a lógica modal proposicional K, bem
      como a realização de testes e avaliação dos algoritmos implementados.
  \end{resumo}

  \selectlanguage{american}
  \begin{abstract}
      Automatic generation of models is a complementary technique to that of
      proof theorems and held in parallel with the experimental evaluation. If
      the program fails to find the proof, the automatically extracted model
      serves as witness of the impossibility of finding such proof. Furthermore,
      with the possibility of combining some strategies, not finding a model
      serves as witness of the incompleteness of such combination, thus being a
      tool of theoretical support. This work aims to implement an automatic
      generator models for propositional modal logic K, as well as testing and
      evaluation of the implemented algorithms.
  \end{abstract}
  \selectlanguage{brazil}

  \tableofcontents
  \listoffigures
  \listoftables

  \textual
  \chapter{Introdução}
\label{cap:intro}

Linguagens lógicas podem ser utilizadas para verificar propriedades de um
sistema complexo, expressas através de possibilidade, crença, probabilidade
entre outras modalidades~\cite{FHMV95,rao:91c,HMM83,Hai82}. Diferentes
modalidades definem diferentes lógicas modais.

Um teorema é uma fórmula que pode ser provada com base nos axiomas e no conjunto
de regras de inferência que definem um determinado cálculo. Uma vez que um
sistema tenha sido especificado na linguagem lógica por meio de fórmulas, é
possível utilizar métodos de provas para verificá-los.

Em~\cite{nalon} são apresentados cálculos baseados em resolução para quinze
famílias de lógicas modais. As regras de inferência baseiam-se nas propriedades
dos modelos subjacentes, ao invés de se fixar na forma dos axiomas. Deste modo,
obtém-se um procedimento uniforme para se lidar com várias lógicas. Uma das
intenções de tal proposta é justamente prover técnicas que facilitem o projeto
de cálculos combinados tanto para fusões de lógicas quanto para lógicas em que
interações fossem permitidas. Interações são, em geral, caracterizadas por
axiomas contendo operadores das diferentes lógicas componentes.

Grande parte dos provadores para lógicas modais são, porém, baseados em
tradução, o que acaba, por vezes, se tornando inconveniente ao usuário. Além
disso, dada a natureza das aplicações descritas com o auxílio destas lógicas, é
normal o uso da combinação de diferentes linguagens modais. A combinação de
linguagens, todavia, pode acarretar no aumento da complexidade ou mesmo na
indecidibilidade do problema de satisfatibilidade na lógica
resultante~\cite{mdml}.
Portanto, é importante o desenvolvimento de técnicas que possam ser utilizadas
de modo uniforme na combinação de métodos de prova para lógicas obtidas a partir
de fusões e/ou em linguagens que permitam interações.

Os métodos apresentados em~\cite{nalon} e em trabalhos anteriores têm esta
característica de uniformidade, mas carecem de refinamentos a fim de permitir a
construção de ferramentas que possam ser, de fato, utilizadas na verificação
formal de sistemas complexos.

O problema básico de satisfatibilidade da lógica modal K é
PSPACE~\cite{complex_modal}. 
Além disso, sabe-se que métodos de prova para lógica proposicional são
intratáveis~\cite{complex_thproving}. Em~\cite{DBLP:conf/tableaux/NalonHD15},
uma estratégia de busca por provas, baseada em níveis modais foi apresentada. A
implementação e a avaliação desta estratégia, comparando resultados de
eficiência com provadores no estado da arte, mostram que a separação em níveis
modais pode auxiliar na melhoria do tempo para a obtenção de provas
\cite{Nalon2016}.

A geração automática de modelos é complementar àquela da prova de teoremas e
realizada em paralelo com a avaliação experimental. Se o provador de teoremas
falha em encontrar uma prova, o modelo automaticamente extraído serve como
testemunha da impossibilidade de se encontrar tal prova. Além disso, com a
possibilidade de uso combinado de estratégias, a não obtenção de um modelo serve
como testemunha da incompletude de tal combinação, sendo portanto ferramenta de
suporte ao entendimento teórico.

O objetivo específico deste trabalho consiste na implementação de um gerador
automático de modelos para a lógica modal proposicional K. A entrada será o
conjunto de cláusulas fornecido pelo provador implementado
em~\cite{Nalon2016}. 
A saída será a declaração da inexistência de um modelo, no caso do conjunto de
cláusulas ser insatisfatível, ou a apresentação formal de um modelo que
testemunhe a satisfatibilidade do conjunto de cláusulas.

Os dados obtidos na experimentação prática serão analisados, em conformidade com
o enfoque corrente, de forma comparativa a partir de \textit{benchmarks}
estabelecidos e especialmente preparados para tal fim. Tais resultados são
apresentados no Capítulo~\ref{cap:resultados}.











  \chapter{Computação quântica}

texto.... referência~\cite{chomsky57}









  \chapter{Construção}
\label{cap:construcao}


  \chapter{Resultados}
\label{cap:resultados}

  % ...

  \postextual
  \bibliographystyle{plain}
  \bibliography{arw12,abbreviations,aiml04,knowledge,logic,mine,modal,ours,p3,theorem-proving,theory,tl}

\end{document}
