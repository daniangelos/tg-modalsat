%%%%%%%%%%%%%%%%%%%%%%%%%%%%%%%%%%%%%%%%
% Classe do documento
%%%%%%%%%%%%%%%%%%%%%%%%%%%%%%%%%%%%%%%%

% Nós usamos a classe "unb-cic".  Deixe apenas uma das linhas
% abaixo não-comentada, dependendo se você for do bacharelado ou
% da licenciatura.

\documentclass[bacharelado]{unb-cic}
%\documentclass[licenciatura]{unb-cic}



%%%%%%%%%%%%%%%%%%%%%%%%%%%%%%%%%%%%%%%%
% Pacotes importados
%%%%%%%%%%%%%%%%%%%%%%%%%%%%%%%%%%%%%%%%

\usepackage[brazil,american]{babel}
\usepackage[T1]{fontenc}
\usepackage{indentfirst}
\usepackage{natbib}
\usepackage{xcolor,graphicx,url}
\usepackage[utf8]{inputenc}



%%%%%%%%%%%%%%%%%%%%%%%%%%%%%%%%%%%%%%%%
% Cores dos links
%%%%%%%%%%%%%%%%%%%%%%%%%%%%%%%%%%%%%%%%

% Veja o arquivos cores.tex se quiser ver que outras cores estão
% pré-definidas.  Utilizando o comando \hypersetup abaixo nós
% evitamos aquelas caixas vermelhas feias em volta dos links.

%%%%%%%%%%%%%%%%%%%%%%%%%%%%%%%%%%%%%%%%
% Cores do estilo Tango
%%%%%%%%%%%%%%%%%%%%%%%%%%%%%%%%%%%%%%%%

\definecolor{LightButter}{rgb}{0.98,0.91,0.31}
\definecolor{LightOrange}{rgb}{0.98,0.68,0.24}
\definecolor{LightChocolate}{rgb}{0.91,0.72,0.43}
\definecolor{LightChameleon}{rgb}{0.54,0.88,0.20}
\definecolor{LightSkyBlue}{rgb}{0.45,0.62,0.81}
\definecolor{LightPlum}{rgb}{0.68,0.50,0.66}
\definecolor{LightScarletRed}{rgb}{0.93,0.16,0.16}
\definecolor{Butter}{rgb}{0.93,0.86,0.25}
\definecolor{Orange}{rgb}{0.96,0.47,0.00}
\definecolor{Chocolate}{rgb}{0.75,0.49,0.07}
\definecolor{Chameleon}{rgb}{0.45,0.82,0.09}
\definecolor{SkyBlue}{rgb}{0.20,0.39,0.64}
\definecolor{Plum}{rgb}{0.46,0.31,0.48}
\definecolor{ScarletRed}{rgb}{0.80,0.00,0.00}
\definecolor{DarkButter}{rgb}{0.77,0.62,0.00}
\definecolor{DarkOrange}{rgb}{0.80,0.36,0.00}
\definecolor{DarkChocolate}{rgb}{0.56,0.35,0.01}
\definecolor{DarkChameleon}{rgb}{0.30,0.60,0.02}
\definecolor{DarkSkyBlue}{rgb}{0.12,0.29,0.53}
\definecolor{DarkPlum}{rgb}{0.36,0.21,0.40}
\definecolor{DarkScarletRed}{rgb}{0.64,0.00,0.00}
\definecolor{Aluminium1}{rgb}{0.93,0.93,0.92}
\definecolor{Aluminium2}{rgb}{0.82,0.84,0.81}
\definecolor{Aluminium3}{rgb}{0.73,0.74,0.71}
\definecolor{Aluminium4}{rgb}{0.53,0.54,0.52}
\definecolor{Aluminium5}{rgb}{0.33,0.34,0.32}
\definecolor{Aluminium6}{rgb}{0.18,0.20,0.21}

\hypersetup{
  colorlinks=true,
  linkcolor=DarkScarletRed,
  citecolor=DarkScarletRed,
  filecolor=DarkScarletRed,
  urlcolor= DarkScarletRed
}



%%%%%%%%%%%%%%%%%%%%%%%%%%%%%%%%%%%%%%%%
% Informações sobre a monografia
%%%%%%%%%%%%%%%%%%%%%%%%%%%%%%%%%%%%%%%%

\title{Título da monografia}

\orientador{\prof \dr Orientador}{CIC/UnB}
%\coorientador[a]{\prof[a] \dr[a] Coorientadora}{MAT/UnB}
\coordenador{\prof \dr Coordenador}{CIC/UnB}
\diamesano{12}{dezembro}{2012}

\membrobanca{\prof \dr Professor I}{CIC/UnB}
\membrobanca{\prof \dr Professor II}{CIC/UnB}

\autor{Nome Sobrenome}{UltimoNome}
\CDU{004.4}

\palavraschave{palvrachave1, palvrachave2, palvrachave3 }
\keywords{keyword1, keyword2, keyword3}



%%%%%%%%%%%%%%%%%%%%%%%%%%%%%%%%%%%%%%%%
% Texto
%%%%%%%%%%%%%%%%%%%%%%%%%%%%%%%%%%%%%%%%

\begin{document}
  \maketitle
  \pretextual

  \begin{dedicatoria}
  Dedico a....
  \end{dedicatoria}

  \begin{agradecimentos}
  Agradeço a....
  \end{agradecimentos}

  \begin{resumo}
  A ciência...
  \end{resumo}

  \selectlanguage{american}
  \begin{abstract}
  The science...
  \end{abstract}
  \selectlanguage{brazil}

  \tableofcontents
  \listoffigures
  \listoftables

  \textual
  \chapter{Introdução}
\label{cap:intro}

Linguagens lógicas podem ser utilizadas para verificar propriedades de um
sistema complexo, expressas através de possibilidade, crença, probabilidade
entre outras modalidades~\cite{FHMV95,rao:91c,HMM83,Hai82}. Diferentes
modalidades definem diferentes lógicas modais.

Um teorema é uma fórmula que pode ser provada com base nos axiomas e no conjunto
de regras de inferência que definem um determinado cálculo. Uma vez que um
sistema tenha sido especificado na linguagem lógica por meio de fórmulas, é
possível utilizar métodos de provas para verificá-los.

Em~\cite{nalon} são apresentados cálculos baseados em resolução para quinze
famílias de lógicas modais. As regras de inferência baseiam-se nas propriedades
dos modelos subjacentes, ao invés de se fixar na forma dos axiomas. Deste modo,
obtém-se um procedimento uniforme para se lidar com várias lógicas. Uma das
intenções de tal proposta é justamente prover técnicas que facilitem o projeto
de cálculos combinados tanto para fusões de lógicas quanto para lógicas em que
interações fossem permitidas. Interações são, em geral, caracterizadas por
axiomas contendo operadores das diferentes lógicas componentes.

Grande parte dos provadores para lógicas modais são, porém, baseados em
tradução, o que acaba, por vezes, se tornando inconveniente ao usuário. Além
disso, dada a natureza das aplicações descritas com o auxílio destas lógicas, é
normal o uso da combinação de diferentes linguagens modais. A combinação de
linguagens, todavia, pode acarretar no aumento da complexidade ou mesmo na
indecidibilidade do problema de satisfatibilidade na lógica
resultante~\cite{mdml}.
Portanto, é importante o desenvolvimento de técnicas que possam ser utilizadas
de modo uniforme na combinação de métodos de prova para lógicas obtidas a partir
de fusões e/ou em linguagens que permitam interações.

Os métodos apresentados em~\cite{nalon} e em trabalhos anteriores têm esta
característica de uniformidade, mas carecem de refinamentos a fim de permitir a
construção de ferramentas que possam ser, de fato, utilizadas na verificação
formal de sistemas complexos.

O problema básico de satisfatibilidade da lógica modal K é
PSPACE~\cite{complex_modal}. 
Além disso, sabe-se que métodos de prova para lógica proposicional são
intratáveis~\cite{complex_thproving}. Em~\cite{DBLP:conf/tableaux/NalonHD15},
uma estratégia de busca por provas, baseada em níveis modais foi apresentada. A
implementação e a avaliação desta estratégia, comparando resultados de
eficiência com provadores no estado da arte, mostram que a separação em níveis
modais pode auxiliar na melhoria do tempo para a obtenção de provas
\cite{Nalon2016}.

A geração automática de modelos é complementar àquela da prova de teoremas e
realizada em paralelo com a avaliação experimental. Se o provador de teoremas
falha em encontrar uma prova, o modelo automaticamente extraído serve como
testemunha da impossibilidade de se encontrar tal prova. Além disso, com a
possibilidade de uso combinado de estratégias, a não obtenção de um modelo serve
como testemunha da incompletude de tal combinação, sendo portanto ferramenta de
suporte ao entendimento teórico.

O objetivo específico deste trabalho consiste na implementação de um gerador
automático de modelos para a lógica modal proposicional K. A entrada será o
conjunto de cláusulas fornecido pelo provador implementado
em~\cite{Nalon2016}. 
A saída será a declaração da inexistência de um modelo, no caso do conjunto de
cláusulas ser insatisfatível, ou a apresentação formal de um modelo que
testemunhe a satisfatibilidade do conjunto de cláusulas.

Os dados obtidos na experimentação prática serão analisados, em conformidade com
o enfoque corrente, de forma comparativa a partir de \textit{benchmarks}
estabelecidos e especialmente preparados para tal fim. Tais resultados são
apresentados no Capítulo~\ref{cap:resultados}.











  \chapter{Computação quântica}

texto.... referência~\cite{chomsky57}









  % ...

  \postextual
  \bibliographystyle{plain}
  \bibliography{bibliografia}

\end{document}
